\documentclass[spanish,12pt]{article}
\date{\today}
\usepackage[letterpaper,margin=1in]{geometry}
\usepackage{ragged2e}
\usepackage{fontspec}
\defaultfontfeatures{Ligatures=TeX,Scale=MatchLowercase}
\setmainfont{IBM Plex Sans}
\setmonofont{IBM Plex Mono}

\usepackage{babel}
\usepackage{translator}
\usepackage{graphicx}
\usepackage{pdflscape}
\usepackage[dvipsnames]{xcolor}
\usepackage{adjustbox}
\usepackage{sepfootnotes}

% Header/Footer
%\usepackage[headsepline,footsepline]{scrlayer-scrpage}
%\clearpairofpagestyles
%\ihead*{\title}
%\ohead*{\thepage}
%\addtokomafont{pageheadfoot}{\upshape}
%\deftripstyle{ChapterStyle}{}{}{}{}{\pagemark}{}
%\renewcommand*{\chapterpagestyle}{ChapterStyle}

% Cite
%\usepackage{citation-style-language}
%\cslsetup{style=apa}
%\addbibresource{bib.bib}

% GANTT
\usepackage{pgfgantt}

% User Stories
\usepackage{.resources/sty/postit}

% MD
\providecommand{\tightlist}{\setlength{\itemsep}{0pt}\setlength{\parskip}{0pt}}

% Portada
\usepackage{tikz}
\usetikzlibrary{calc}
\usepackage{anyfontsize}

% Style
\AddToHook{cmd/section/before}{\clearpage} % New page with each section
\setlength{\parindent}{0pt}
\setlength{\parskip}{6pt plus 2pt minus 1pt}
\setlength{\emergencystretch}{3em}

\usepackage{enumitem}
\setlist{nolistsep}
\setlength\itemsep{0pt}
\setlength{\itemindent}{0pt}
\setlength{\parsep}{0pt}

\begin{document}
\justifying
\pagestyle{empty}

\begin{tikzpicture}[overlay,remember picture]

\fill[black!2] (current page.south west) rectangle (current page.north east);

\begin{scope}[transform canvas ={rotate around ={45:($(current page.north west)+(-.5,-6)$)}}]

\shade[rounded corners=18pt, left color=Dandelion, right color=Dandelion!40] ($(current page.north west)+(-.5,-6)$) rectangle ++(9,1.5);

\end{scope}

\begin{scope}[transform canvas ={rotate around ={45:($(current page.north west)+(.5,-10)$)}}]

\shade[rounded corners=18pt, left color=lightgray,right color=lightgray!60] ($(current page.north west)+(0.5,-10)$) rectangle ++(15,1.5);

\end{scope}

\begin{scope}[transform canvas ={rotate around ={45:($(current page.north west)+(0.5,-10)$)}}]

\shade[rounded corners=8pt, left color=lightgray] ($(current page.north west)+(1.5,-9.55)$) rectangle ++(7,.6);

\end{scope}

\begin{scope}[transform canvas ={rotate around ={45:($(current page.north)+(-1.5,-3)$)}}]

\shade[rounded corners=12pt, left color=orange!80, right color=orange!60] ($(current page.north)+(-1.5,-3)$) rectangle ++(9,0.8);

\end{scope}

\begin{scope}[transform canvas ={rotate around ={45:($(current page.north)+(-3,-8)$)}}]

\shade[rounded corners=28pt, left color=red!80, right color=red!80] ($(current page.north)+(-3,-8)$) rectangle ++(15,1.8);

\end{scope}

\begin{scope}[transform canvas ={rotate around ={45:($(current page.north west)+(4,-15.5)$)}}]

\shade[rounded corners=25pt, left color=orange, right color=Dandelion] ($(current page.north west)+(4,-15.5)$) rectangle ++(30,1.8);

\end{scope}

\begin{scope}[transform canvas ={rotate around ={45:($(current page.north west)+(13,-10)$)}},]

\shade[rounded corners=22pt, left color=RoyalBlue

,right color=Emerald] ($(current page.north west)+(13,-10)$) rectangle ++(15,1.5);

\end{scope}

\begin{scope}[transform canvas ={rotate around ={45:($(current page.north west)+(18,-8)$)}},]

\shade[rounded corners=8pt, left color=lightgray] ($(current page.north west)+(18,-8)$) rectangle ++(15,0.6);

\end{scope}

\begin{scope}[transform canvas ={rotate around ={45:($(current page.north west)+(19,-5.65)$)}},]

\shade[rounded corners=12pt, left color=lightgray] ($(current page.north west)+(19,-5.65)$) rectangle ++(15,0.8);

\end{scope}

\begin{scope}[transform canvas ={rotate around ={45:($(current page.north west)+(20,-9)$)}}]

\shade[rounded corners=20pt, left color=OrangeRed, right color=red!80] ($(current page.north west)+(20,-9)$) rectangle ++(14,1.2);

\end{scope}

\draw[ultra thick,gray] ($(current page.center)+(5,2)$) -- ++(0,-3cm) node[midway,left=0.25cm,text width=5cm,align=right,black!75]{{\fontsize{25}{30} \selectfont \bf GESTIÓN DE \\[10pt] GIMNASIOS}} node[midway,right=0.25cm,text width=6cm,align=left,orange]{{\fontsize{72}{86.4} \selectfont 2024}};

\node at ($(current page.center)+(0,-4)$) {{\fontsize{60}{72} \selectfont Sistema Distribuido}};

\node[text width=\linewidth,align=center] at ($(current page.center)+(0,-8)$) {{\fontsize{16}{19.2} \selectfont \textcolor{orange}{ \bf SQL-Slayers}} \\[3pt] Barón Salinas Mauricio Valentín\\ Cristóbal Silverio Cristian\\ Gómez Daniel Aimar Jair \\ Hernández Reyes Magaly\\ Sánchez Carrasco Monserrat\\ Ramírez Suárez Gerardo};

\end{tikzpicture}
\tableofcontents
\section{Definición de Proyecto}\label{definiciuxf3n-de-proyecto}

\subsection{Introducción}\label{introducciuxf3n}

\subsection{Planteamiento del
problema}\label{planteamiento-del-problema}

\subsection{Objetivo General}\label{objetivo-general}

\subsection{Objetivos Especificos}\label{objetivos-especificos}

\subsection{Justificación}\label{justificaciuxf3n}

\subsection{Alcances y Limitaciones}\label{alcances-y-limitaciones}

\subsubsection{Alcances del Proyecto}\label{alcances-del-proyecto}

\begin{enumerate}
\def\labelenumi{\arabic{enumi}.}
\tightlist
\item
  \textbf{Distribución Geográfica de Datos}: Implementación de una base
  de datos distribuida que permita almacenar y gestionar datos en
  múltiples ubicaciones físicas ---en nuestro caso, las sedes de
  gimnasio gestionadas por el cliente--- para mejorar la disponibilidad
  y la redundancia.
\item
  \textbf{Sincronización y Consistencia de Datos}: Desarrollo de
  mecanismos que garantizen que los datos sean consistentes entre las
  distintas ubicaciones y que cualquier actualización se propague
  adecuadamente.
\item
  \textbf{Escalabilidad}: Diseño de la base de datos para que pueda
  escalar horizontalmente, añadiendo más nodos según sea necesario para
  manejar el aumento del volumen de datos y la carga de trabajo.
\item
  \textbf{Seguridad de Datos}: Implementación de medidas de seguridad
  que protejan los datos distribuidos contra accesos no autorizados y
  posibles ataques cibernéticos.
\item
  \textbf{Resiliencia y Recuperación ante Desastres}: Establecemiento de
  procedimientos y tecnologías para asegurar la recuperación rápida y
  efectiva en caso de fallos o desastres.
\end{enumerate}

\subsubsection{Limitaciones del
Proyecto}\label{limitaciones-del-proyecto}

\begin{enumerate}
\def\labelenumi{\arabic{enumi}.}
\tightlist
\item
  \textbf{Latencia de Red}: La latencia de la red entre los nodos puede
  afectar el rendimiento del sistema, especialmente si los nodos están
  ubicados en regiones geográficas distantes.
\item
  \textbf{Complejidad de Gestión}: La administración y el mantenimiento
  de una base de datos distribuida pueden ser significativamente más
  complejos en comparación con una base de datos centralizada.
\item
  \textbf{Costos de Infraestructura}: Los costos asociados con el
  hardware, el software y la red para soportar una base de datos
  distribuida pueden ser elevados.
\item
  \textbf{Consistencia Eventual}: Algunas arquitecturas distribuidas
  optan por la consistencia eventual, lo que puede llevar a periodos en
  los que los datos no están sincronizados en todos los nodos.
\item
  \textbf{Problemas de Sincronización}: La sincronización de datos entre
  nodos puede enfrentar problemas, especialmente en redes con alta
  latencia o en escenarios de particionamiento de red.
\end{enumerate}

\section{Marco teorico}\label{marco-teorico}

\subsection{Descripción}\label{descripciuxf3n}

\subsubsection{Metodologia de Analisis}\label{metodologia-de-analisis}

\paragraph{¿Qué es metodología
Scrum?}\label{quuxe9-es-metodologuxeda-scrum}

Es un pequeño equipo de personas, un equipo Scrum. El equipo Scrum
consta de: - un Scrum Master, - un propietario de producto (Product
Owner) y - desarrolladores.

Dentro de un equipo de Scrum, no hay sub-equipos ni jerarquías. Es una
unidad cohesionada de profesionales enfocada en un objetivo a la vez, el
objetivo del Producto.

Scrum (nombre que proviene de cierta jugada que tiene lugar durante un
partido de rugby) es un método de desarrollo ágil de software concebido
por Jeff Sutherland y su equipo de desarrollo a principios de la década
de 1990.

Retraso: lista de prioridades de los requerimientos o características
del proyecto que dan al cliente un valor del negocio. Es posible agregar
en cualquier momento otros aspectos al retraso (ésta es la forma en la
que se introducen los cambios). El gerente del proyecto evalúa el
retraso y actualiza las prioridades según se requiera. Sprints: consiste
en unidades de trabajo que se necesitan para alcanzar un requerimiento
definido en el retraso que debe ajustarse en una caja de tiempo
predefinida (lo común son 30 días). Durante el sprint no se introducen
cambios (por ejemplo, aspectos del trabajo retrasado). Así, el sprint
permite a los miembros del equipo trabajar en un ambiente de corto plazo
pero estable. Reuniones Scrum: son reuniones breves (de 15 minutos, por
lo general) que el equipo Scrum efectúa a diario. Hay tres preguntas
clave que se pide que respondan todos los miembros del equipo. • ¿Qué
hiciste desde la última reunión del equipo? • ¿Qué obstáculos estás
encontrando? • ¿Qué planeas hacer mientras llega la siguiente reunión
del equipo? Un líder del equipo, llamado maestro Scrum, dirige la junta
y evalúa las respuestas de cada persona.

\paragraph{¿Qué es una Metodología de
análisis?}\label{quuxe9-es-una-metodologuxeda-de-anuxe1lisis}

Nosotros enfocamos un marco teorico que influye en el proceso
estructurado para guiar el desarrollo desde los elementos de
entrada(información recopilada), hasta las herramientas que se
utilizaran para la elaboración de ellas. Asi facilitando su
implementación en las diferentes etapas del proceso. Pressman promueve
un enfoque iterativo e incrementalen e desarrollo de software, en este
caso lo utilizaremos en el desarrollo de nuestra base de datos
distribuida en base al proyecto del GYM, esto para que se contruya y
mejore partes del proyecto de manera gradual. Esto tambien tomando en
cuenta los aspectos de gestión de proyectos, estimación, planificación y
control, asi tambien la detección temprana de los defectos y su
corrección inmediata.

\section{Historias de usuario y Casos de
uso}\label{historias-de-usuario-y-casos-de-uso}

\subsection{Casos de Uso}\label{casos-de-uso}

Analiza el comportamiento del sistema en distintas condiciones en las
que el sistema responde a una petición de alguno de sus participantes.
En esencia, un caso de uso narra una historia estilizada sobre cómo
interactúa un usuario final (que tiene cierto número de roles posibles)
con el sistema en circunstancias específicas. Un caso de uso ilustra el
software o sistema desde el punto de vista del usuario final.

\subsection{Historias de Usuario}\label{historias-de-usuario}

Las historia de usuario consta de ser un texto narrativo, un lineamiento
de tareas o interacciones, una descripción basada en un formato o una
representación diagramática. Sin importar su forma

\subsection{Entrevista de
requerimientos}\label{entrevista-de-requerimientos}

Éste es el enfoque más directo, los miembros del equipo de software se
reúnen con los usuarios para entender mejor sus necesidades,
motivaciones, cultura laboral y una multitud de aspectos adicionales.
Esto se logra en reuniones individuales o a través de grupos de enfoque.

\subsection{Analisis}\label{analisis}

\subsubsection{Entrevista}\label{entrevista}

\textbf{Entrevistador:} Buenas tardes. Gracias por tomarse el tiempo
para esta entrevista. Para empezar, ¿puede darme una visión general de
su negocio y qué es lo que espera lograr con este nuevo sistema?

\textbf{Administrador:} Buenas tardes. Claro, tengo varios gimnasios en
diferentes ubicaciones, y actualmente gestionamos todo manualmente o con
sistemas muy básicos. Lo que realmente necesito es un sistema que me
permita gestionar las inscripciones de los miembros, el seguimiento de
su actividad y productividad, y también la administración de los
empleados y horarios.

\textbf{Entrevistador:} Entiendo. Vamos a desglosar esto en partes.
Primero, ¿puede describir cómo maneja actualmente el registro de
miembros y qué aspectos le gustaría mejorar?

\textbf{Administrador:} En este momento, usamos un sistema de hojas de
cálculo para el registro de nuevos miembros y el seguimiento de pagos.
Esto es bastante ineficiente y propenso a errores. Me gustaría que el
nuevo sistema permitiera registrar automáticamente nuevos miembros y
gestionar sus pagos.

\subsubsection{Registro de miembros}\label{registro-de-miembros}

\textbf{Entrevistador:} Perfecto. Para el registro de miembros, ¿qué
información específica necesita capturar?

\textbf{Administrador:} Necesitamos capturar información básica como:

\begin{itemize}
\tightlist
\item
  nombre,
\item
  número de teléfono principal y
\item
  número de teléfono alternativo.
\end{itemize}

También necesitamos registrar:

\begin{itemize}
\tightlist
\item
  el tipo de membresía,
\item
  la fecha de inicio y
\item
  los pagos realizados.
\end{itemize}

\subsubsection{Seguimiento
Administrativo}\label{seguimiento-administrativo}

\textbf{Entrevistador:} En relación a la administración de empleados y
horarios, ¿qué funcionalidades específicas está buscando?

\textbf{Administrador:} Necesito un módulo para gestionar:

\begin{itemize}
\tightlist
\item
  los horarios de los empleados, incluyendo:

  \begin{itemize}
  \tightlist
  \item
    las asignaciones de turno y
  \item
    las solicitudes de vacaciones.
  \end{itemize}
\end{itemize}

También sería útil poder hacer un seguimiento de:

\begin{itemize}
\tightlist
\item
  sus horas trabajadas y
\item
  sus pagos.
\end{itemize}

\subsubsection{Accesos al sistema}\label{accesos-al-sistema}

\textbf{Entrevistador:} Entiendo. ¿Qué tipo de acceso y permisos
deberían tener los diferentes usuarios del sistema? Por ejemplo, ¿qué
pueden ver o hacer los administradores frente a los empleados o los
entrenadores?

\textbf{Administrador:}

\begin{itemize}
\tightlist
\item
  Los administradores deberían tener acceso completo a todas las
  funciones del sistema, incluyendo: la gestión de miembros, empleados,
  informes y configuraciones.
\item
  Los empleados y entrenadores solo deberían tener acceso a funciones
  relacionadas con su trabajo, como ver sus horarios y gestionar sus
  propias clases.
\end{itemize}

\subsubsection{Integración}\label{integraciuxf3n}

\textbf{Entrevistador:} Muy bien. En términos de integración, ¿hay otros
sistemas o herramientas con los que el nuevo sistema debería integrarse,
como sistemas de contabilidad o plataformas de marketing?

\textbf{Administrador:} Sí, el sistema debería integrarse con nuestro
software de contabilidad para que los pagos se registren
automáticamente.

\subsubsection{Soporte y mantenimiento}\label{soporte-y-mantenimiento}

\textbf{Entrevistador:} Perfecto. Para finalizar, ¿hay algún otro
requerimiento o aspecto específico que no hayamos cubierto y que sea
importante para usted?

\textbf{Administrador:} Creo que lo hemos cubierto bastante bien. Solo
me gustaría:

\begin{itemize}
\tightlist
\item
  que el sistema fuera fácil de usar,
\item
  que tuviera un buen soporte técnico y
\item
  una interfaz llamativa.
\end{itemize}

La facilidad de uso es importante para que tanto los empleados como los
miembros puedan adaptarse rápidamente.

\subsubsection{Últimas palabras}\label{uxfaltimas-palabras}

\textbf{Entrevistador:} Entendido. Muchas gracias por la información
detallada. Con estos requisitos, podemos empezar a definir el alcance
del proyecto y trabajar en el diseño del sistema. ¿Hay algo más que
quiera agregar antes de que terminemos?

\textbf{Administrador:} No, eso es todo. Gracias a usted.

\textbf{Entrevistador:} De nada. Le mantendremos informado sobre el
avance del proyecto. Que tenga un buen día.

\textbf{Administrador:} Igualmente. Que tenga un buen día.

\subsection{Historias de Usuario}\label{historias-de-usuario-1}

\begin{itemize}
\tightlist
\item
  Diagramas de Casos de uso
\item
  Historias de Usuario
\end{itemize}

\subsection{Diseño}\label{diseuxf1o}

\begin{itemize}
\tightlist
\item
  Arquitectura de Sistema de Base de Datos Distriuido
\item
  Esquema Conceptual Global
\item
  Esquema Conceptual Local
\end{itemize}

\subsection{Desarrollo}\label{desarrollo}

\begin{itemize}
\tightlist
\item
  Implementacion del SBDD
\item
  Implementacion De Interfaces de Usuario
\item
  Pruebas Unitarias
\end{itemize}

\subsection{Prioridad de historias de
usuario}\label{prioridad-de-historias-de-usuario}
\end{document}
