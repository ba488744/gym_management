\documentclass[spanish, 12pt]{article}
\date{\today}
\usepackage[letterpaper, margin=1in]{geometry}
\usepackage{ragged2e}
\usepackage{fontspec}
\defaultfontfeatures{Ligatures=TeX,Scale=MatchLowercase}
\setmainfont{IBM Plex Sans}
\setmonofont{IBM Plex Mono}

\usepackage{babel}
\usepackage{translator}
\usepackage{graphicx}
\usepackage{pdflscape}
\usepackage[dvipsnames]{xcolor}
\usepackage{adjustbox}
\usepackage{sepfootnotes}

% Header/Footer
%\usepackage[headsepline,footsepline]{scrlayer-scrpage}
%\clearpairofpagestyles
%\ihead*{\title}
%\ohead*{\thepage}
%\addtokomafont{pageheadfoot}{\upshape}
%\deftripstyle{ChapterStyle}{}{}{}{}{\pagemark}{}
%\renewcommand*{\chapterpagestyle}{ChapterStyle}

% Cite
%\usepackage{citation-style-language}
%\cslsetup{style=apa}
%\addbibresource{bib.bib}

% GANTT
\usepackage{pgfgantt}

% User Stories
\usepackage{.resources/sty/postit}

% MD
\providecommand{\tightlist}{\setlength{\itemsep}{0pt}
\setlength{\parskip}{0pt}}

% Portada
\usepackage{tikz}
\usetikzlibrary{calc}
\usepackage{anyfontsize}

% Style
\AddToHook{cmd/section/before}{\clearpage} % New page with each section
\setlength{\parindent}{0pt}
\setlength{\parskip}{6pt plus 2pt minus 1pt}
\setlength{\emergencystretch}{3em}

\usepackage{enumitem}
\setlist{nolistsep}
\setlength{\itemsep}{0pt}
\setlength{\itemindent}{0pt}
\setlength{\parsep}{0pt}

\begin{document}
	\justifying
	\pagestyle{empty}

	\begin{tikzpicture}[overlay, remember picture]
		\fill[black!2]
			(current page.south west) rectangle (current page.north east);

		\begin{scope}[
			transform canvas={rotate around ={45:($(current page.north west)+(-.5,-6)$)}}
		]
			\shade[
				rounded corners=18pt,
				left color=Dandelion,
				right color=Dandelion!40
			] ($(current page.north west)+(-.5,-6)$) rectangle ++(9,1.5);
		\end{scope}

		\begin{scope}[
			transform canvas={rotate around ={45:($(current page.north west)+(.5,-10)$)}}
		]
			\shade[
				rounded corners=18pt,
				left color=lightgray,
				right color=lightgray!60
			] ($(current page.north west)+(0.5,-10)$) rectangle ++(15,1.5);
		\end{scope}

		\begin{scope}[
			transform canvas={rotate around ={45:($(current page.north west)+(0.5,-10)$)}}
		]
			\shade[rounded corners=8pt, left color=lightgray]
				($(current page.north west)+(1.5,-9.55)$) rectangle ++(7,.6);
		\end{scope}

		\begin{scope}[
			transform canvas={rotate around ={45:($(current page.north)+(-1.5,-3)$)}}
		]
			\shade[rounded corners=12pt, left color=orange!80, right color=orange!60]
				($(current page.north)+(-1.5,-3)$) rectangle ++(9,0.8);
		\end{scope}

		\begin{scope}[
			transform canvas={rotate around ={45:($(current page.north)+(-3,-8)$)}}
		]
			\shade[rounded corners=28pt, left color=red!80, right color=red!80]
				($(current page.north)+(-3,-8)$) rectangle ++(15,1.8);
		\end{scope}

		\begin{scope}[
			transform canvas={rotate around ={45:($(current page.north west)+(4,-15.5)$)}}
		]
			\shade[rounded corners=25pt, left color=orange, right color=Dandelion]
				($(current page.north west)+(4,-15.5)$) rectangle ++(30,1.8);
		\end{scope}

		\begin{scope}[
			transform canvas={rotate around ={45:($(current page.north west)+(13,-10)$)}},
		]
			\shade[rounded corners=22pt, left color=RoyalBlue, right color=Emerald]
				($(current page.north west)+(13,-10)$) rectangle ++(15,1.5);
		\end{scope}

		\begin{scope}[
			transform canvas={rotate around ={45:($(current page.north west)+(18,-8)$)}},
		]
			\shade[rounded corners=8pt, left color=lightgray]
				($(current page.north west)+(18,-8)$) rectangle ++(15,0.6);
		\end{scope}

		\begin{scope}[
			transform canvas={rotate around ={45:($(current page.north west)+(19,-5.65)$)}},
		]
			\shade[rounded corners=12pt, left color=lightgray]
				($(current page.north west)+(19,-5.65)$) rectangle ++(15,0.8);
		\end{scope}

		\begin{scope}[
			transform canvas={rotate around ={45:($(current page.north west)+(20,-9)$)}}
		]
			\shade[rounded corners=20pt, left color=OrangeRed, right color=red!80]
				($(current page.north west)+(20,-9)$) rectangle ++(14,1.2);
		\end{scope}

		\draw[ultra thick, gray]
			($(current page.center)+(5,2)$) -- ++(0,-3cm)
			node[midway, left=0.25cm, text width=5cm, align=right, black!75]
				{{\fontsize{25}{30} \selectfont \bf GESTIÓN DE \\[10pt] GIMNASIOS}}
			node[midway, right=0.25cm, text width=6cm, align=left, orange]
				{{\fontsize{72}{86.4} \selectfont 2024}};

		\node at
			($(current page.center)+(0,-4)$)
			{{\fontsize{60}{72} \selectfont Sistema Distribuido}};

		\node[text width=\linewidth, align=center]
			at
			($(current page.center)+(0,-8)$)
			{{\fontsize{16}{19.2} \selectfont \textcolor{orange}{ \bf SQL-Slayers}} \\[3pt] Barón Salinas Mauricio Valentín\\ Cristóbal Silverio Cristian\\ Gómez Daniel Aimar Jair \\ Hernández Reyes Magaly\\ Sánchez Carrasco Monserrat\\ Ramírez Suárez Gerardo};
	\end{tikzpicture}
	\tableofcontents
	\section{Introduccion}
	\label{introduccion}

	\section{Marco teorico}
	\label{marco-teorico}

	\subsection{Descripcion}
	\label{descripcion}

	\subsection{Entrevista de Requisitos: Administrador de Gimnasios}
	\label{entrevista-de-requisitos-administrador-de-gimnasios}

	\subsubsection{Visión general de negocios}
	\label{visiuxf3n-general-de-negocios}

	\textbf{Entrevistador:} Buenas tardes. Gracias por tomarse el tiempo para esta
	entrevista. Para empezar, ¿puede darme una visión general de su negocio y qué es
	lo que espera lograr con este nuevo sistema?

	\textbf{Administrador:} Buenas tardes. Claro, tengo varios gimnasios en
	diferentes ubicaciones, y actualmente gestionamos todo manualmente o con sistemas
	muy básicos. Lo que realmente necesito es un sistema que me permita gestionar
	las inscripciones de los miembros, el seguimiento de su actividad y productividad,
	y también la administración de los empleados y horarios.

	\textbf{Entrevistador:} Entiendo. Vamos a desglosar esto en partes. Primero,
	¿puede describir cómo maneja actualmente el registro de miembros y qué
	aspectos le gustaría mejorar?

	\textbf{Administrador:} En este momento, usamos un sistema de hojas de cálculo
	para el registro de nuevos miembros y el seguimiento de pagos. Esto es bastante
	ineficiente y propenso a errores. Me gustaría que el nuevo sistema permitiera
	registrar automáticamente nuevos miembros, gestionar sus pagos y proporcionar informes
	detallados sobre su actividad y asistencia.

	\subsubsection{Registro de miembros}
	\label{registro-de-miembros}

	\textbf{Entrevistador:} Perfecto. Para el registro de miembros, ¿qué información
	específica necesita capturar?

	\textbf{Administrador:} Necesitamos capturar información básica como:

	\begin{itemize}
		\tightlist

		\item nombre,

		\item número de teléfono principal y

		\item número de teléfono alternativo.
	\end{itemize}

	También necesitamos registrar:

	\begin{itemize}
		\tightlist

		\item el tipo de membresía,

		\item la fecha de inicio y

		\item los pagos realizados.
	\end{itemize}

	Además, me gustaría poder tener acceso a:

	\begin{itemize}
		\tightlist

		\item un historial de actividad del miembro.
	\end{itemize}

	\subsubsection{Seguimiento Administrativo}
	\label{seguimiento-administrativo}

	\textbf{Entrevistador:} Muy bien. En cuanto al seguimiento de la productividad,
	¿qué tipo de métricas o indicadores le interesan?

	\textbf{Administrador:} Me gustaría tener información sobre:

	\begin{itemize}
		\tightlist

		\item la asistencia de los miembros y

		\item la frecuencia con la que utilizan las instalaciones.
	\end{itemize}

	También sería útil conocer:

	\begin{itemize}
		\tightlist

		\item el nivel de satisfacción del cliente a través de encuestas integradas en
			el sistema.
	\end{itemize}

	\subsubsection{Recursos Humanos}
	\label{recursos-humanos}

	\textbf{Entrevistador:} En relación a la administración de empleados y horarios,
	¿qué funcionalidades específicas está buscando?

	\textbf{Administrador:} Necesito un módulo para gestionar:

	\begin{itemize}
		\tightlist

		\item los horarios de los empleados, incluyendo:

			\begin{itemize}
				\tightlist

				\item las asignaciones de turno y

				\item las solicitudes de vacaciones.
			\end{itemize}
	\end{itemize}

	También sería útil poder hacer un seguimiento de:

	\begin{itemize}
		\tightlist

		\item sus horas trabajadas y

		\item sus pagos.
	\end{itemize}

	\subsubsection{Accesos al sistema}
	\label{accesos-al-sistema}

	\textbf{Entrevistador:} Entiendo. ¿Qué tipo de acceso y permisos deberían
	tener los diferentes usuarios del sistema? Por ejemplo, ¿qué pueden ver o
	hacer los administradores frente a los empleados o los entrenadores?

	\textbf{Administrador:}

	\begin{itemize}
		\tightlist

		\item Los administradores deberían tener acceso completo a todas las funciones
			del sistema, incluyendo: la gestión de miembros, empleados, informes y configuraciones.

		\item Los empleados y entrenadores solo deberían tener acceso a funciones relacionadas
			con su trabajo, como ver sus horarios y gestionar sus propias clases.

		\item Los miembros deberían tener acceso a sus propios datos, horarios de
			clases y la posibilidad de reservar sesiones.
	\end{itemize}

	\subsubsection{Integración}
	\label{integraciuxf3n}

	\textbf{Entrevistador:} Muy bien. En términos de integración, ¿hay otros sistemas
	o herramientas con los que el nuevo sistema debería integrarse, como sistemas
	de contabilidad o plataformas de marketing?

	\textbf{Administrador:} Sí, el sistema debería integrarse con nuestro software
	de contabilidad para que los pagos se registren automáticamente.

	\subsubsection{Soporte y mantenimiento}
	\label{soporte-y-mantenimiento}

	\textbf{Entrevistador:} Perfecto. Para finalizar, ¿hay algún otro
	requerimiento o aspecto específico que no hayamos cubierto y que sea
	importante para usted?

	\textbf{Administrador:} Creo que lo hemos cubierto bastante bien. Solo me
	gustaría:

	\begin{itemize}
		\tightlist

		\item que el sistema fuera fácil de usar y

		\item que tuviera un buen soporte técnico.
	\end{itemize}

	La facilidad de uso es importante para que tanto los empleados como los miembros
	puedan adaptarse rápidamente.

	\subsubsection{Últimas palabras}
	\label{uxfaltimas-palabras}

	\textbf{Entrevistador:} Entendido. Muchas gracias por la información detallada.
	Con estos requisitos, podemos empezar a definir el alcance del proyecto y
	trabajar en el diseño del sistema. ¿Hay algo más que quiera agregar antes de que
	terminemos?

	\textbf{Administrador:} No, eso es todo. Gracias a usted.

	\textbf{Entrevistador:} De nada. Le mantendremos informado sobre el avance del
	proyecto. Que tenga un buen día.

	\textbf{Administrador:} Igualmente. Que tenga un buen día.

	\subsection{Historias de Usuario}
	\label{historias-de-usuario}

	\subsubsection{Registro de Miembros}
	\label{registro-de-miembros-1}

	Como \textbf{administrador}, Quiero \textbf{poder registrar nuevos miembros en
	el sistema}, Para \textbf{automatizar el proceso de inscripción y tener un
	registro preciso de los datos de los miembros}.

	\textbf{Criterios de Aceptación:}

	\begin{itemize}
		\tightlist

		\item El sistema debe permitir ingresar información básica del miembro:
			nombre, número de teléfono principal y número de teléfono alternativo.

		\item El sistema debe registrar el tipo de membresía, fecha de inicio y
			pagos realizados.

		\item El sistema debe mantener un historial de actividad del miembro.
	\end{itemize}

	\subsubsection{Gestión de Pagos}
	\label{gestiuxf3n-de-pagos}

	Como \textbf{administrador}, Quiero \textbf{poder gestionar los pagos de los
	miembros dentro del sistema}, Para \textbf{asegurarme de que los pagos se
	registren de manera precisa y automática}.

	\textbf{Criterios de Aceptación:}

	\begin{itemize}
		\tightlist

		\item El sistema debe registrar pagos realizados por los miembros.

		\item El sistema debe permitir consultar el estado de cuenta y el historial de
			pagos.
	\end{itemize}

	\subsubsection{Seguimiento Administrativo}
	\label{seguimiento-administrativo-1}

	Como \textbf{administrador}, Quiero \textbf{tener acceso a informes sobre la
	asistencia y frecuencia de uso de las instalaciones por parte de los miembros},
	Para \textbf{monitorear la utilización de las instalaciones y mejorar la
	gestión de recursos}.

	\textbf{Criterios de Aceptación:}

	\begin{itemize}
		\tightlist

		\item El sistema debe generar informes sobre la asistencia de los miembros.

		\item El sistema debe mostrar la frecuencia con la que los miembros utilizan
			las instalaciones y asisten a clases.

		\item El sistema debe permitir la integración de encuestas para medir la satisfacción
			del cliente.
	\end{itemize}

	\subsubsection{Gestión de Empleados y Horarios}
	\label{gestiuxf3n-de-empleados-y-horarios}

	Como \textbf{administrador}, Quiero \textbf{poder gestionar los horarios de
	los empleados, incluyendo turnos y solicitudes de vacaciones}, Para \textbf{facilitar
	la administración del personal y asegurar una cobertura adecuada en los
	gimnasios}.

	\textbf{Criterios de Aceptación:}

	\begin{itemize}
		\tightlist

		\item El sistema debe permitir asignar turnos y gestionar solicitudes de vacaciones.

		\item El sistema debe hacer seguimiento de las horas trabajadas por cada empleado.

		\item El sistema debe gestionar los pagos de los empleados.
	\end{itemize}

	\subsubsection{Control de Acceso y Permisos}
	\label{control-de-acceso-y-permisos}

	Como \textbf{administrador}, Quiero \textbf{definir diferentes niveles de
	acceso y permisos para los usuarios del sistema}, Para \textbf{asegurar que
	cada usuario tenga acceso solo a las funciones relevantes para su rol}.

	\textbf{Criterios de Aceptación:}

	\begin{itemize}
		\tightlist

		\item El sistema debe permitir asignar roles y permisos específicos a administradores,
			empleados, entrenadores y miembros.

		\item Los administradores deben tener acceso completo a todas las funciones del
			sistema.

		\item Los empleados y entrenadores deben tener acceso solo a las funciones relacionadas
			con su trabajo.

		\item Los miembros deben tener acceso a sus propios datos y reservas.
	\end{itemize}

	\subsubsection{Integración con Otros Sistemas}
	\label{integraciuxf3n-con-otros-sistemas}

	Como \textbf{administrador}, Quiero \textbf{integrar el sistema con el
	software de contabilidad y plataformas de marketing}, Para \textbf{facilitar
	la gestión financiera y la comunicación con los miembros}.

	\textbf{Criterios de Aceptación:}

	\begin{itemize}
		\tightlist

		\item El sistema debe integrarse con el software de contabilidad para registrar
			automáticamente los pagos.

		\item El sistema debe permitir la integración con plataformas de marketing para
			enviar campañas de correo electrónico y promociones.
	\end{itemize}

	\subsubsection{Facilidad de Uso y Soporte Técnico}
	\label{facilidad-de-uso-y-soporte-tuxe9cnico}

	Como \textbf{usuario del sistema}, Quiero \textbf{que el sistema sea fácil de
	usar y cuente con buen soporte técnico}, Para \textbf{poder adaptarme
	rápidamente y recibir ayuda cuando sea necesario}.

	\textbf{Criterios de Aceptación:}

	\begin{itemize}
		\tightlist

		\item El sistema debe tener una interfaz intuitiva y fácil de usar.

		\item Debe estar disponible soporte técnico para resolver problemas y responder
			preguntas.
	\end{itemize}

	\subsection{Prioridad de historias de usuario}
	\label{prioridad-de-historias-de-usuario}
\end{document}