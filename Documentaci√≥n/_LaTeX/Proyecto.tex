\documentclass[spanish, 12pt]{article}
\usepackage[letterpaper, margin=1in]{geometry}
\usepackage[spanish]{babel}
\usepackage{translator}
\usepackage[justify]{ragged2e}
\usepackage{setspace}
\setstretch{1.2}
\usepackage{fontspec}
\defaultfontfeatures{Ligatures=TeX,Scale=MatchLowercase}
\setmainfont{IBM Plex Sans}
\setmonofont{IBM Plex Mono}
\usepackage{graphicx}
\usepackage{pdflscape}
\usepackage[dvipsnames]{xcolor}
\usepackage{adjustbox}
\usepackage{sepfootnotes}

% Header/Footer
%\usepackage[headsepline,footsepline]{scrlayer-scrpage}
%\clearpairofpagestyles
%\ihead*{\title}
%\ohead*{\thepage}
%\addtokomafont{pageheadfoot}{\upshape}
%\deftripstyle{ChapterStyle}{}{}{}{}{\pagemark}{}
%\renewcommand*{\chapterpagestyle}{ChapterStyle}

% Cite
%\usepackage{citation-style-language}
%\cslsetup{style=apa}
%\addbibresource{bib.bib}
%\usepackage{csquotes}

% GANTT
\usepackage{pgfgantt}

% User Stories
\usepackage{.resources/sty/postit}

% MD
\providecommand{\tightlist}{\setlength{\itemsep}{0pt}
\setlength{\parskip}{0pt}}

% Portada
\usepackage{tikz}
\usetikzlibrary{calc}
\usepackage{anyfontsize}

% Style
\AddToHook{cmd/section/before}{\clearpage} % Nueva página en cada sección
\setlength{\parindent}{0pt} %
\setlength{\parskip}{6pt plus 2pt minus 1pt} %
\setlength{\emergencystretch}{3em} %

% Pretty Footnotes
\usepackage[hang, flushmargin, bottom, multiple]{footmisc}
\setlength{\footnotemargin}{0.8em} % set space between footnote nr and text
\setlength{\footnotesep}{\baselineskip} % set space between multiple footnotes
\setlength{\skip\footins}{0.3cm} % set space between page content and footnote
\setlength{\footskip}{0.9cm}

% Blockquote
\definecolor{blockquote-border}{RGB}{221,221,221}
\definecolor{blockquote-text}{RGB}{119,119,119}
\usepackage{mdframed}
\newmdenv[rightline=false,bottomline=false,topline=false,linewidth=3pt,linecolor=blockquote-border,skipabove=\parskip]{customblockquote}
\renewenvironment{quote}{\begin{customblockquote}\list{}{\rightmargin=0em\leftmargin=0em}%
\item\relax
\color{blockquote-text}
\ignorespaces}{\unskip\unskip\endlist\end{customblockquote}}

% Enumerations and lists
\usepackage{enumitem}
\setlist{nolistsep}
\setlength{\itemsep}{0pt}
\setlength{\itemindent}{0pt}
\setlength{\parsep}{0pt}
\begin{document}
	\pagestyle{empty}

	\begin{tikzpicture}[overlay, remember picture]
		\fill[black!2]
			(current page.south west) rectangle (current page.north east);

		\begin{scope}[
			transform canvas={rotate around ={45:($(current page.north west)+(-.5,-6)$)}}
		]
			\shade[
				rounded corners=18pt,
				left color=Dandelion,
				right color=Dandelion!40
			] ($(current page.north west)+(-.5,-6)$) rectangle ++(9,1.5);
		\end{scope}

		\begin{scope}[
			transform canvas={rotate around ={45:($(current page.north west)+(.5,-10)$)}}
		]
			\shade[
				rounded corners=18pt,
				left color=lightgray,
				right color=lightgray!60
			] ($(current page.north west)+(0.5,-10)$) rectangle ++(15,1.5);
		\end{scope}

		\begin{scope}[
			transform canvas={rotate around ={45:($(current page.north west)+(0.5,-10)$)}}
		]
			\shade[rounded corners=8pt, left color=lightgray]
				($(current page.north west)+(1.5,-9.55)$) rectangle ++(7,.6);
		\end{scope}

		\begin{scope}[
			transform canvas={rotate around ={45:($(current page.north)+(-1.5,-3)$)}}
		]
			\shade[rounded corners=12pt, left color=orange!80, right color=orange!60]
				($(current page.north)+(-1.5,-3)$) rectangle ++(9,0.8);
		\end{scope}

		\begin{scope}[
			transform canvas={rotate around ={45:($(current page.north)+(-3,-8)$)}}
		]
			\shade[rounded corners=28pt, left color=red!80, right color=red!80]
				($(current page.north)+(-3,-8)$) rectangle ++(15,1.8);
		\end{scope}

		\begin{scope}[
			transform canvas={rotate around ={45:($(current page.north west)+(4,-15.5)$)}}
		]
			\shade[rounded corners=25pt, left color=orange, right color=Dandelion]
				($(current page.north west)+(4,-15.5)$) rectangle ++(30,1.8);
		\end{scope}

		\begin{scope}[
			transform canvas={rotate around ={45:($(current page.north west)+(13,-10)$)}},
		]
			\shade[rounded corners=22pt, left color=RoyalBlue, right color=Emerald]
				($(current page.north west)+(13,-10)$) rectangle ++(15,1.5);
		\end{scope}

		\begin{scope}[
			transform canvas={rotate around ={45:($(current page.north west)+(18,-8)$)}},
		]
			\shade[rounded corners=8pt, left color=lightgray]
				($(current page.north west)+(18,-8)$) rectangle ++(15,0.6);
		\end{scope}

		\begin{scope}[
			transform canvas={rotate around ={45:($(current page.north west)+(19,-5.65)$)}},
		]
			\shade[rounded corners=12pt, left color=lightgray]
				($(current page.north west)+(19,-5.65)$) rectangle ++(15,0.8);
		\end{scope}

		\begin{scope}[
			transform canvas={rotate around ={45:($(current page.north west)+(20,-9)$)}}
		]
			\shade[rounded corners=20pt, left color=OrangeRed, right color=red!80]
				($(current page.north west)+(20,-9)$) rectangle ++(14,1.2);
		\end{scope}

		\draw[ultra thick, gray]
			($(current page.center)+(5,2)$) -- ++(0,-3cm)
			node[midway, left=0.25cm, text width=5cm, align=right, black!75]
				{{\fontsize{25}{30} \selectfont \bf GESTIÓN DE \\[10pt] GIMNASIOS}}
			node[midway, right=0.25cm, text width=6cm, align=left, orange]
				{{\fontsize{72}{86.4} \selectfont 2024}};

		\node at
			($(current page.center)+(0,-4)$)
			{{\fontsize{60}{72} \selectfont Sistema Distribuido}};

		\node[text width=\linewidth, align=center]
			at
			($(current page.center)+(0,-8)$)
			{{\fontsize{16}{19.2} \selectfont \textcolor{orange}{ \bf SQL-Slayers}} \\[3pt] Barón Salinas Mauricio Valentín\\ Cristóbal Silverio Cristian\\ Gómez Daniel Aimar Jair \\ Hernández Reyes Magaly\\ Sánchez Carrasco Monserrat\\ Ramírez Suárez Gerardo};
	\end{tikzpicture}
	\tableofcontents
	\section{Definición de Proyecto}
	\label{definiciuxf3n-de-proyecto}

	El proyecto en cuestión plantea el desarrollo de una base de datos para gestionar
	de manera eficiente las operaciones de una serie de gimnasios. Su enfoque
	principal está en la optimización de la administración diaria y en la mejora
	de la experiencia de los usuarios a través de un sistema que maneje de manera
	integral la información de membresías, pagos, empleados y el inventario de
	productos o equipos. El propósito final de este sistema es garantizar que todas
	las áreas operativas del gimnasio sean gestionadas de manera eficiente, con la
	capacidad de escalar a medida que el negocio crezca.

	Se destacan cinco objetivos específicos que buscan cubrir aspectos claves del
	funcionamiento del gimnasio:

	\begin{enumerate}
		\def\labelenumi{\arabic{enumi}.} \tightlist

		\item \textbf{Distribución geográfica} de los datos para gestionar múltiples
			sedes.

		\item \textbf{Sincronización y consistencia} de la información entre esas
			sedes.

		\item \textbf{Escalabilidad} del sistema para manejar el crecimiento del
			negocio.

		\item \textbf{Implementación} de medidas de seguridad para proteger los
			datos.

		\item \textbf{Garantizar la resiliencia y recuperación} en caso de desastres.
	\end{enumerate}

	A su vez, el proyecto identifica algunas limitaciones, tales como la latencia de
	la red entre las distintas ubicaciones, la complejidad en la gestión de un
	sistema distribuido, los costos de infraestructura asociados, y posibles
	problemas de sincronización en redes de alta latencia.

	\subsection{Introducción}
	\label{introducciuxf3n}

	En la actualidad, la eficiencia de un gimnasio depende no solo de la calidad
	de sus instalaciones o de su personal, si no también de la correcta gestión de
	los datos de sus miembros, participantes y operaciones diarias. Los gimnasios de
	hoy en dia manejan una enorme cantidad de información que contiene el control de
	membresías, el seguimiento de sus pagos e información personal de sus
	integrantes. Por ello, sin un sistema bien organizado es fácil perder el
	control sobre los aspectos antes mencionados y ocasiona un experiencia
	negativa en los usuarios.

	Este proyecto tiene la tarea de crear una base de datos que permita la
	excelente gestión de un gimnasio para optimizar la administración diaria, mejorar
	el servicio al cliente y aumentar la eficiencia operativa.

	\subsection{Planteamiento del problema}
	\label{planteamiento-del-problema}

	El gimnasio enfrenta dificultades para gestionar eficientemente las membresías,
	pagos y datos de usuarios. La falta de un sistema integral provoca
	desorganización, pérdida de información y dificultades en el seguimiento de las
	transacciones, lo que puede afectar la experiencia del cliente y la
	rentabilidad del negocio.

	\subsection{Objetivo General}
	\label{objetivo-general}

	Desarrollar una base de datos para una serie de gimnasios, buscando una
	capacidad de escalabilidad, que nos permita optimizar la administración y el control
	de todas las áreas operativas, incluyendo gestión de membresías, el manejo de inventario
	de equipos y productos como suplemnetos de igual manera el seguimiento de los empleados
	de cada sucursal.

	La base de datos nos dara la integridad y disponibilidad de la informacion en tiempo
	real para facilitar la toma de decisiones estratégicas, mejorar la eficiencia
	operativa y asegurar un control efectivo de los recursos a medida que el
	negocio crezca.

	\subsection{Objetivos Especificos}
	\label{objetivos-especificos}

	\begin{itemize}
		\tightlist

		\item Desarrollar una base de datos distribuida que permita el almacenamiento
			de la información de los clientes, entrenadores, rutinas de entrenamiento,
			y pagos.

		\item Garantizar la comunicación entre los nodos de la base de datos para que
			la información esté sincronizada y disponible en todo momento,
			independientemente de la localización geográfica de los gimnasios o
			sucursales.

		\item Desarrollar una interfaz de usuario amigable, utilizando HTML, CSS y JavaScript,
			que permita a los usuarios acceder, visualizar y modificar datos.
	\end{itemize}

	\subsection{Justificación}
	\label{justificaciuxf3n}

	La necesidad de desarrollar una base de datos robusta para la gestión de un gimnasio
	surge de la creciente complejidad en la administración de los datos relacionados
	con los usuarios, sus membresías, los pagos y el manejo de inventario. En el
	contexto actual, donde la digitalización de los servicios se ha convertido en un
	estándar, contar con un sistema de gestión que permita optimizar las
	operaciones diarias no solo es una ventaja competitiva, sino también una necesidad
	fundamental para la sostenibilidad y crecimiento de un gimnasio.

	Uno de los problemas principales que enfrentan los gimnasios es la desorganización
	y falta de control sobre los datos de sus miembros y transacciones financieras.
	La ausencia de un sistema centralizado de información puede dar lugar a la
	pérdida de datos, dificultades en la trazabilidad de las operaciones y errores
	en la gestión de inventarios, lo que puede perjudicar la experiencia del
	cliente y reducir la rentabilidad del negocio.

	Por lo tanto, la implementación de una base de datos no solo resolvería estos problemas,
	sino que también ofrecería una serie de beneficios adicionales. Por ejemplo,
	permitiría un acceso rápido y seguro a la información en tiempo real, lo que facilitaría
	la toma de decisiones informadas. Además, la posibilidad de contar con un sistema
	que distribuya los datos geográficamente en múltiples ubicaciones permitiría gestionar
	de manera efectiva las distintas sedes de un gimnasio, asegurando que toda la
	información esté sincronizada y disponible para los administradores y usuarios
	en cualquier momento y lugar.

	Asimismo, la escalabilidad del sistema es un aspecto crítico en este proyecto,
	ya que a medida que el gimnasio crezca y aumente la cantidad de miembros, empleados
	y equipos, la base de datos debe ser capaz de manejar este crecimiento sin
	comprometer la eficiencia operativa. La seguridad también es un factor esencial,
	especialmente cuando se trata de la protección de información personal y
	financiera de los usuarios. Las medidas de seguridad que se implementarán estarán
	destinadas a prevenir accesos no autorizados y garantizar la integridad de los
	datos.

	Otro punto a considerar es la resiliencia del sistema. Dado que un gimnasio
	maneja información crítica de sus operaciones, es indispensable que la base de
	datos esté preparada para recuperar la información de manera rápida y efectiva
	en caso de cualquier eventualidad, como fallos técnicos o desastres naturales.
	Esto asegurará la continuidad de las operaciones sin afectar el servicio al cliente.

	\subsection{Alcances y Limitaciones}
	\label{alcances-y-limitaciones}

	\subsubsection{Alcances}
	\label{alcances}

	\begin{enumerate}
		\def\labelenumi{\arabic{enumi}.} \tightlist

		\item \textbf{Distribución Geográfica de Datos}: Implementación de una base de
			datos distribuida que permita almacenar y gestionar datos en múltiples ubicaciones
			físicas ---en nuestro caso, las sedes de gimnasio gestionadas por el cliente---
			para mejorar la disponibilidad y la redundancia.

		\item \textbf{Sincronización y Consistencia de Datos}: Desarrollo de mecanismos
			que garantizen que los datos sean consistentes entre las distintas ubicaciones
			y que cualquier actualización se propague adecuadamente.

		\item \textbf{Escalabilidad}: Diseño de la base de datos para que pueda escalar
			horizontalmente, añadiendo más nodos según sea necesario para manejar el
			aumento del volumen de datos y la carga de trabajo.

		\item \textbf{Seguridad de Datos}: Implementación de medidas de seguridad que
			protejan los datos distribuidos contra accesos no autorizados y posibles ataques
			cibernéticos.

		\item \textbf{Resiliencia y Recuperación ante Desastres}: Establecemiento de
			procedimientos y tecnologías para asegurar la recuperación rápida y efectiva
			en caso de fallos o desastres.
	\end{enumerate}

	\subsubsection{Limitaciones}
	\label{limitaciones}

	\begin{enumerate}
		\def\labelenumi{\arabic{enumi}.} \tightlist

		\item \textbf{Latencia de Red}: La latencia de la red entre los nodos puede afectar
			el rendimiento del sistema, especialmente si los nodos están ubicados en
			regiones geográficas distantes.

		\item \textbf{Complejidad de Gestión}: La administración y el mantenimiento de
			una base de datos distribuida pueden ser significativamente más complejos en
			comparación con una base de datos centralizada.

		\item \textbf{Costos de Infraestructura}: Los costos asociados con el hardware,
			el software y la red para soportar una base de datos distribuida pueden
			ser elevados.

		\item \textbf{Consistencia Eventual}: Algunas arquitecturas distribuidas optan
			por la consistencia eventual, lo que puede llevar a periodos en los que
			los datos no están sincronizados en todos los nodos.

		\item \textbf{Problemas de Sincronización}: La sincronización de datos entre
			nodos puede enfrentar problemas, especialmente en redes con alta latencia
			o en escenarios de particionamiento de red.
	\end{enumerate}

	\section{Marco teórico}
	\label{marco-teuxf3rico}

	\subsection{Metodología de Análisis}
	\label{metodologuxeda-de-anuxe1lisis}

	\subsubsection{SCRUM}
	\label{scrum}

	SCRUM es un marco ágil para la gestión de proyectos, especialmente en el
	desarrollo de software. Se basa en la colaboración, la autoorganización y la
	entrega incremental de productos. SCRUM utiliza roles específicos (como el
	Product Owner, Scrum Master y el equipo de desarrollo), así como eventos estructurados
	(como sprints, reuniones diarias y revisiones de sprint) para facilitar la
	planificación, ejecución y mejora continua del trabajo. Su objetivo es
	aumentar la flexibilidad y la adaptación a cambios en los requisitos del
	proyecto.

	Un equipo SCRUM es un pequeño equipo de personas, que consta de:

	\begin{itemize}
		\tightlist

		\item un Scrum Master,

		\item un propietario de producto (Product Owner) y

		\item desarrolladores.
	\end{itemize}

	Dentro de un equipo de Scrum, no hay sub-equipos ni jerarquías. Es una unidad
	cohesionada de profesionales enfocada en un objetivo a la vez, el objetivo del
	Producto.

	Retraso: lista de prioridades de los requerimientos o características del
	proyecto que dan al cliente un valor del negocio. Es posible agregar en cualquier
	momento otros aspectos al retraso (ésta es la forma en la que se introducen
	los cambios). El gerente del proyecto evalúa el retraso y actualiza las
	prioridades según se requiera. Sprints: consiste en unidades de trabajo que se
	necesitan para alcanzar un requerimiento definido en el retraso que debe
	ajustarse en una caja de tiempo predefinida (lo común son 30 días). Durante el
	sprint no se introducen cambios (por ejemplo, aspectos del trabajo retrasado).
	Así, el sprint permite a los miembros del equipo trabajar en un ambiente de
	corto plazo pero estable. Reuniones Scrum: son reuniones breves (de 15 minutos,
	por lo general) que el equipo Scrum efectúa a diario. Hay tres preguntas clave
	que se pide que respondan todos los miembros del equipo.

	\begin{itemize}
		\tightlist

		\item ¿Qué hiciste desde la última reunión del equipo?

		\item ¿Qué obstáculos estás encontrando?

		\item ¿Qué planeas hacer mientras llega la siguiente reunión del equipo? Un
			líder del equipo, llamado maestro Scrum, dirige la junta y evalúa las
			respuestas de cada persona.
	\end{itemize}

	\subsubsection{¿Qué es una Metodología de análisis?}
	\label{quuxe9-es-una-metodologuxeda-de-anuxe1lisis}

	Nosotros enfocamos un marco teorico que influye en el proceso estructurado
	para guiar el desarrollo desde los elementos de entrada(información recopilada),
	hasta las herramientas que se utilizaran para la elaboración de ellas. Asi facilitando
	su implementación en las diferentes etapas del proceso. Pressman promueve un
	enfoque iterativo e incrementalen e desarrollo de software, en este caso lo utilizaremos
	en el desarrollo de nuestra base de datos distribuida en base al proyecto del GYM,
	esto para que se contruya y mejore partes del proyecto de manera gradual. Esto
	tambien tomando en cuenta los aspectos de gestión de proyectos, estimación, planificación
	y control, asi tambien la detección temprana de los defectos y su corrección
	inmediata.

	\subsubsection{Casos de Uso}
	\label{casos-de-uso}

	Analiza el comportamiento del sistema en distintas condiciones en las que el sistema
	responde a una petición de alguno de sus participantes. En esencia, un caso de
	uso narra una historia estilizada sobre cómo interactúa un usuario final (que
	tiene cierto número de roles posibles) con el sistema en circunstancias específicas.
	Un caso de uso ilustra el software o sistema desde el punto de vista del
	usuario final.

	\subsubsection{Historias de Usuario}
	\label{historias-de-usuario}

	Las historia de usuario consta de ser un texto narrativo, un lineamiento de
	tareas o interacciones, una descripción basada en un formato o una representación
	diagramática. Sin importar su forma

	\subsubsection{Entrevista de requerimientos}
	\label{entrevista-de-requerimientos}

	Éste es el enfoque más directo, los miembros del equipo de software se reúnen con
	los usuarios para entender mejor sus necesidades, motivaciones, cultura laboral
	y una multitud de aspectos adicionales. Esto se logra en reuniones
	individuales o a través de grupos de enfoque.

	\subsection{Metodología de Diseño}
	\label{metodologuxeda-de-diseuxf1o}

	Una metodología de diseño se compone de varios elementos que guían el proceso creativo
	y la resolución de problemas. Una correcta metodología de diseño dentro de los
	procesos de desarrollo debe tener ciertas características y secuencia para que
	su salida consista en un activo potencial para el desarrollo del sistema.

	\subsubsection{Definición del Problema}
	\label{definiciuxf3n-del-problema}

	\begin{itemize}
		\tightlist

		\item Definición clara del problema a abordar

		\item Identificación de las necesidades y expectativas de los usuarios

		\item Definición de los objetivos del diseño
	\end{itemize}

	\subsubsection{Desarrollo de Conceptos}
	\label{desarrollo-de-conceptos}

	\begin{itemize}
		\tightlist

		\item Proposición de ideas.

		\item Selección de las ideas más prometedoras

		\item Desarrollo en conceptos más concretos.

		\item Esbozo de prototipos iniciales o maquetas para visualizar y evaluar las
			soluciones propuestas.
	\end{itemize}

	\subsubsection{Evaluación y Pruebas}
	\label{evaluaciuxf3n-y-pruebas}

	\begin{itemize}
		\tightlist

		\item Los conceptos desarrollados se prueban y evalúan, a menudo mediante
			feedback de usuarios.

		\item Se identifican fallos y áreas de mejora, permitiendo iteraciones en el
			diseño.
	\end{itemize}

	\subsection{Metodología de Desarrollo}
	\label{metodologuxeda-de-desarrollo}

	Una metodología de desarrollo proporciona un marco estructurado para planificar,
	ejecutar y gestionar proyectos de manera eficiente. A continuación, se
	describen brevemente sus componentes fundamentales:

	\subsubsection{Planificación}
	\label{planificaciuxf3n}

	\begin{itemize}
		\tightlist

		\item Definición del alcance del proyecto

		\item Identificación de los objetivos

		\item Elaboración de un cronograma

		\item Definición de los recursos necesarios

		\item Asignación de roles y responsabilidades al equipo
	\end{itemize}

	\subsubsection{Pruebas}
	\label{pruebas}

	\begin{itemize}
		\tightlist

		\item Desarrollo de pruebas para identificar errores y garantizar que el producto
			cumple con los requisitos establecidos, asegurando que el software sea
			robusto y confiable:

			\begin{itemize}
				\tightlist

				\item Pruebas unitarias,

				\item de integración y

				\item de aceptación del usuario.
			\end{itemize}
	\end{itemize}

	\subsubsection{Implementación}
	\label{implementaciuxf3n}

	Una vez que el software ha sido probado y validado, se despliega en el entorno
	de producción. Esto puede incluir la migración de datos y la capacitación de usuarios,
	asegurando una transición fluida.

	\subsubsection{Mantenimiento}
	\label{mantenimiento}

	Después de la implementación, se requiere un seguimiento continuo para
	corregir errores, realizar actualizaciones y mejorar el sistema en función del
	feedback de los usuarios. Este componente es esencial para asegurar la
	longevidad y la relevancia del producto.

	\subsubsection{Evaluación y Mejora Continua}
	\label{evaluaciuxf3n-y-mejora-continua}

	Se lleva a cabo una revisión del proceso y los resultados del proyecto. Esta
	evaluación permite identificar lecciones aprendidas y oportunidades de mejora,
	lo que contribuye a la optimización de futuras metodologías de desarrollo.

	\section{Análisis}
	\label{anuxe1lisis}

	\subsection{Definición del Alcance del Proyecto}
	\label{definiciuxf3n-del-alcance-del-proyecto}

	\begin{itemize}
		\tightlist

		\item \textbf{Objetivo}: Crear un sistema que gestione las operaciones de un
			gimnasio, incluyendo el registro de miembros, gestión de pagos, control
			administrativo de empleados y reportes de ganancias e inventario de
			suplementos.

		\item \textbf{Partes interesadas}: Administradores, miembros y empleados.
	\end{itemize}

	\subsection{Entrevista con el Administrador}
	\label{entrevista-con-el-administrador}

	\subsubsection{Visión de negocio}
	\label{visiuxf3n-de-negocio}

	\textbf{Entrevistador:} Buenas tardes. Gracias por tomarse el tiempo para esta
	entrevista. Para empezar, ¿puede darme una visión general de su negocio y qué
	es lo que espera lograr con este nuevo sistema?

	\textbf{Administrador:} Buenas tardes. Claro, tengo varios gimnasios en diferentes
	ubicaciones, y actualmente gestionamos todo manualmente o con sistemas muy
	básicos. Lo que realmente necesito es un sistema que me permita gestionar las inscripciones
	de los miembros, el seguimiento de su actividad y productividad, y también la administración
	de los empleados y horarios.

	\textbf{Entrevistador:} Entiendo. Vamos a desglosar esto en partes. Primero, ¿puede
	describir cómo maneja actualmente el registro de miembros y qué aspectos le gustaría
	mejorar?

	\textbf{Administrador:} En este momento, usamos un sistema de hojas de cálculo
	para el registro de nuevos miembros y el seguimiento de pagos. Esto es
	bastante ineficiente y propenso a errores. Me gustaría que el nuevo sistema permitiera
	registrar automáticamente nuevos miembros y gestionar sus pagos.

	\subsubsection{Registro de miembros y pagos}
	\label{registro-de-miembros-y-pagos}

	\textbf{Entrevistador:} Perfecto. Para el registro de miembros, ¿qué información
	específica necesita capturar?

	\textbf{Administrador:} Necesitamos capturar información básica como:

	\begin{itemize}
		\tightlist

		\item nombre,

		\item número de teléfono principal y

		\item número de teléfono alternativo.
	\end{itemize}

	También necesitamos registrar:

	\begin{itemize}
		\tightlist

		\item el tipo de membresía,

		\item la fecha de inicio y

		\item los pagos realizados.
	\end{itemize}

	\subsubsection{Seguimiento Administrativo}
	\label{seguimiento-administrativo}

	\textbf{Entrevistador:} En relación a la administración de empleados y
	horarios, ¿qué funcionalidades específicas está buscando?

	\textbf{Administrador:} Necesito un módulo para gestionar:

	\begin{itemize}
		\tightlist

		\item los horarios de los empleados, incluyendo:

			\begin{itemize}
				\tightlist

				\item las asignaciones de turno y

				\item las solicitudes de vacaciones.
			\end{itemize}
	\end{itemize}

	También sería útil poder hacer un seguimiento de:

	\begin{itemize}
		\tightlist

		\item sus horas trabajadas y

		\item sus pagos.
	\end{itemize}

	\subsubsection{Gestión de suplementos}
	\label{gestiuxf3n-de-suplementos}

	\textbf{Entrevistador:} Perfecto. Para finalizar, ¿hay algún otro
	requerimiento o aspecto específico que no hayamos cubierto y que sea
	importante para usted?

	\textbf{Administrador:} Sí, actualmente también servimos como distribuidor de
	suplementos, por lo que me gustaría que el sistema me permita:

	\begin{itemize}
		\tightlist

		\item llevar un control de los suplementos vendidos y las ganancias, y

		\item el inventario de los suplementos que tenemos disponibles para venta.
	\end{itemize}

	\subsubsection{Últimas palabras}
	\label{uxfaltimas-palabras}

	\textbf{Entrevistador:} Entendido. Muchas gracias por la información detallada.
	Con estos requisitos, podemos empezar a definir el alcance del proyecto y trabajar
	en el diseño del sistema. ¿Hay algo más que quiera agregar antes de que
	terminemos?

	\textbf{Administrador:} No, eso es todo. Gracias a usted.

	\textbf{Entrevistador:} De nada. Le mantendremos informado sobre el avance del
	proyecto. Que tenga un buen día.

	\textbf{Administrador:} Igualmente. Que tenga un buen día.

	\subsection{Analisis de Requerimientos}
	\label{analisis-de-requerimientos}

	\subsubsection{Funcionales}
	\label{funcionales}

	\begin{itemize}
		\tightlist

		\item \textbf{Gestión de clientes}: Registros, renovación y cancelación de
			membresías.

		\item \textbf{Pagos}: Procesamiento de pagos.

		\item \textbf{Gestión de empleados}: Registros, asisnación de rol y
			actividades a realizar y evaluación de desempeño.

		\item \textbf{Administración del inventario}:
	\end{itemize}

	\subsubsection{No Funcionales}
	\label{no-funcionales}

	\begin{itemize}
		\tightlist

		\item \textbf{Escalabilidad}: Capacidad del sistema para manejar un número creciente
			del usuario.

		\item \textbf{Disponibilidad}: Garantizar que el sistema esté disponible 24/7.

		\item \textbf{Seguridad}: Protección de datos personales y financieros de los
			miembros.

		\item \textbf{Rendimiento}: Respuesta rápida del sistema incluso en horas pico.
	\end{itemize}

	\subsection{Historias de Usuario}
	\label{historias-de-usuario-1}

	\subsubsection{Registro de Miembros}
	\label{registro-de-miembros}

	Como \textbf{administrador}, quiero \textbf{poder registrar nuevos miembros en
	el sistema}, para \textbf{automatizar el proceso de inscripción y tener un
	registro preciso de los datos de los miembros}.

	\begin{itemize}
		\tightlist

		\item El sistema debe permitir ingresar información básica del miembro:
			nombre y número de teléfono

		\item El sistema debe registrar el tipo de membresía, fecha de inicio, fecha
			final y pagos realizados.
	\end{itemize}

	\subsubsection{Gestión de Pagos}
	\label{gestiuxf3n-de-pagos}

	Como \textbf{administrador}, quiero \textbf{poder gestionar los pagos de los
	miembros dentro del sistema}, para \textbf{asegurarme de que los pagos se
	registren de manera precisa y automática}.

	\begin{itemize}
		\tightlist

		\item El sistema debe registrar pagos realizados por los miembros.

		\item El sistema debe permitir consultar el estado de cuenta y el historial de
			pagos.
	\end{itemize}

	\subsubsection{Gestión de Empleados y Horarios}
	\label{gestiuxf3n-de-empleados-y-horarios}

	Como \textbf{administrador}, quiero \textbf{poder gestionar los horarios de
	los empleados, incluyendo turnos}, para \textbf{llevar un orden y supervisar
	las actividades de los trabajadores}.

	\begin{itemize}
		\tightlist

		\item El sistema debe hacer seguimiento de las horas trabajadas por cada empleado.

		\item El sistema debe gestionar los pagos de los empleados.
	\end{itemize}

	\subsubsection{Gestión de inventario de suplementos}
	\label{gestiuxf3n-de-inventario-de-suplementos}

	Como \textbf{administrador}, quiero \textbf{poder llevar un registro de la
	venta e inventario de suplementos} para \textbf{monitorear las ganancias y la
	disponibilidad de éstos}.

	\begin{itemize}
		\tightlist

		\item El sistema debe de permitir controlar el inventario de mis suplementos

		\item El sistema debe registrar las ganancias y precios de los suplementos.
	\end{itemize}

	\section{Diseño}
	\label{diseuxf1o}

	\subsection{Arquitectura de Sistema de Base de Datos Distriuido}
	\label{arquitectura-de-sistema-de-base-de-datos-distriuido}

	\subsection{Esquema Conceptual Global}
	\label{esquema-conceptual-global}

	\subsection{Esquema Conceptual Local}
	\label{esquema-conceptual-local}

	\subsection{Análisis de entidades}
	\label{anuxe1lisis-de-entidades}

	\subsubsection{Entidades y atributos}
	\label{entidades-y-atributos}

	\begin{itemize}
		\tightlist

		\item \textbf{Gimnasio}

			\begin{itemize}
				\tightlist

				\item id\_gimnasio (Llave primaria)

				\item Nombre

				\item Dirección

				\item Teléfono

				\item Horario de Apertura

				\item Horario de Cierre
			\end{itemize}

		\item \textbf{Cliente}

			\begin{itemize}
				\tightlist

				\item id\_cliente (Llave primaria)

				\item Nombre

				\item Apellido Paterno

				\item Apellido Materno

				\item Teléfono

				\item Fecha de Inscripción
			\end{itemize}

		\item \textbf{Membresía}

			\begin{itemize}
				\tightlist

				\item id\_membresía (Llave primaria)

				\item Tipo (Mensual, Anual, etc.)

				\item Precio

				\item Fecha de inicio

				\item Fecha de fin

				\item id\_cliente (Llave foránea)
			\end{itemize}

		\item \textbf{Empleado}

			\begin{itemize}
				\tightlist

				\item id\_empleado (Llave primaria)

				\item Nombre

				\item Apellido Paterno

				\item Apellido Materno

				\item Tipo de empleado

				\item Sueldo

				\item Teléfono
			\end{itemize}
	\end{itemize}

	\subsubsection{Relaciones entre Entidades}
	\label{relaciones-entre-entidades}

	\section{Desarrollo}
	\label{desarrollo}

	\subsection{Implementacion del SBDD}
	\label{implementacion-del-sbdd}

	\subsection{Implementacion De Interfaces de Usuario}
	\label{implementacion-de-interfaces-de-usuario}

	\subsection{Pruebas Unitarias}
	\label{pruebas-unitarias}

	\section{Resultados y conclusiones}
	\label{resultados-y-conclusiones}
	\section{Anexos}
	\label{anexos}
	\subsection{Vision Board}
	\label{vision-board}
	\definecolor{html1}{HTML}{FFFF77}
	\definecolor{html2}{HTML}{FF77FF}
	\definecolor{html3}{HTML}{FF7777}
	\definecolor{html4}{HTML}{77FFFF}
	\definecolor{html5}{HTML}{77FF77}
	\definecolor{html6}{HTML}{7777FF}
	\definecolor{html7}{HTML}{FFBB77}
	\definecolor{html8}{HTML}{FF77BB}
	\definecolor{html9}{HTML}{BBFF77}
	\definecolor{html10}{HTML}{BB77FF}
	\definecolor{html11}{HTML}{77FFBB}
	\definecolor{html12}{HTML}{77BBFF}
	\noindent
	\begin{minipage}{0.498\linewidth}
		\noindent
		\begin{PostItNote}
			[Width=0.9\linewidth,Corner=true,Pin=None,Color=html2,Rotate=-3 ,Title={ Registro de Miembros},FontTitle={\bfseries\itshape}
			] Como \textbf{administrador},

			Quiero \textbf{poder registrar nuevos miembros en el sistema},

			Para \textbf{automatizar el proceso de inscripción y tener un registro
			preciso de los datos de los miembros}.
		\end{PostItNote}
		\vspace{0.5cm}
	\end{minipage}
	\noindent
	\begin{minipage}{0.498\linewidth}
		\noindent
		\begin{PostItNote}
			[Width=0.9\linewidth,Corner=true,Pin=None,Color=html3,Rotate=2.2 ,Title={ Gestión de Pagos},FontTitle={\bfseries\itshape}
			] Como \textbf{administrador},

			Quiero \textbf{poder gestionar los pagos de los miembros dentro del
			sistema},

			Para \textbf{asegurarme de que los pagos se registren de manera precisa y
			automática}.
		\end{PostItNote}
		\vspace{0.5cm}
	\end{minipage}
	\noindent
	\begin{minipage}{0.498\linewidth}
		\noindent
		\begin{PostItNote}
			[Width=0.9\linewidth,Corner=true,Pin=None,Color=html4,Rotate=1.7 ,Title={ Gestión de Empleados y Horarios},FontTitle={\bfseries\itshape}
			] Como \textbf{administrador},

			Quiero \textbf{poder gestionar los horarios de los empleados, incluyendo turnos},

			Para \textbf{facilitar la administración del personal y asegurar una
			cobertura adecuada en los gimnasios}.
		\end{PostItNote}
		\vspace{0.5cm}
	\end{minipage}
	\noindent
	\begin{minipage}{0.498\linewidth}
		\noindent
		\begin{PostItNote}
			[Width=0.9\linewidth,Corner=true,Pin=None,Color=html5,Rotate=-0.3 ,Title={ Control de Acceso y Permisos},FontTitle={\bfseries\itshape}
			] Como \textbf{administrador},

			Quiero \textbf{poder llevar un registro de la venta e inventario de
			suplementos},

			Para \textbf{monitorear las ganancias y la disponibilidad de éstos}.
		\end{PostItNote}
		\vspace{0.5cm}
	\end{minipage}
	\subsection{Cronograma de actividades}
	\label{cronograma}
	\subsubsection{Definición de proyecto y Marco Teórico}

	\hfill
	\begin{adjustbox}
		{max size={0.85\textwidth}{0.85\textheight}} \scalebox{3}{ \def\pgfcalendarweekdayletter#1{%
		\ifcase#1L\or M\or M\or J\or V\or S\or D\fi%
		}
		\begin{ganttchart}
			[ x unit=0.5cm, y unit title=1cm, y unit chart=1cm, time slot format=big-endian,
			vgrid, hgrid, title label font=\footnotesize, progress label text={},
			newline shortcut=true, group label node/.append style={align=right}, bar label
			node/.append style={align=right}, milestone label node/.append style={align=right},
			canvas/.style={ draw=black, dotted } ]{2024-09-12}{2024-09-24}
			\gantttitlecalendar{year, month=name, weekday=letter, day} \\
			\definecolor{html1}{HTML}{FFFF00}
			\definecolor{html2}{HTML}{FF00FF}
			\definecolor{html3}{HTML}{FF0000}
			\definecolor{html4}{HTML}{00FFFF}
			\definecolor{html5}{HTML}{00FF00}
			\definecolor{html6}{HTML}{0000FF}
			\definecolor{html7}{HTML}{FF7700}
			\definecolor{html8}{HTML}{FF0077}
			\definecolor{html9}{HTML}{77FF00}
			\definecolor{html10}{HTML}{7700FF}
			\definecolor{html11}{HTML}{00FF77}
			\definecolor{html12}{HTML}{0077FF}
			\ganttset{ bar/.append style={ pattern color=html1 }, group/.append style={ draw=black, fill=html1 }, milestone/.append style={ fill=html1 }, }
			\ganttset{ bar/.append style={ pattern=dots }, } \ganttgroup[progress=100]{Definición de proyecto}{2024-09-12}{2024-09-17}
			\\ \ganttset{ bar/.append style={ pattern=dots }, } \ganttbar[progress=100]{\raisebox{-0.075cm}{ Introducción } \ganttalignnewline \raisebox{0.15cm}{ \footnotesize{ \textit{ \textcolor{gray}{ Responsable(s): } \textcolor{html1}{ \textbf{ Magaly}} }}}}{2024-09-12}{2024-09-17}
			\\ \ganttset{ bar/.append style={ pattern=crosshatch dots }, } \ganttbar[progress=100]{\raisebox{-0.075cm}{ Planteamiento del problema } \ganttalignnewline \raisebox{0.15cm}{ \footnotesize{ \textit{ \textcolor{gray}{ Responsable(s): } \textcolor{html1}{ \textbf{ Monserrat}} }}}}{2024-09-12}{2024-09-17}
			\\ \ganttset{ bar/.append style={ pattern=horizontal lines }, } \ganttbar[progress=100]{\raisebox{-0.075cm}{ Objetivo general } \ganttalignnewline \raisebox{0.15cm}{ \footnotesize{ \textit{ \textcolor{gray}{ Responsable(s): } \textcolor{html1}{ \textbf{ Cristian}} }}}}{2024-09-12}{2024-09-17}
			\\ \ganttset{ bar/.append style={ pattern=vertical lines }, } \ganttbar[progress=100]{\raisebox{-0.075cm}{ Objetivos específicos } \ganttalignnewline \raisebox{0.15cm}{ \footnotesize{ \textit{ \textcolor{gray}{ Responsable(s): } \textcolor{html1}{ \textbf{ Aimar Jair}} }}}}{2024-09-12}{2024-09-17}
			\\ \ganttset{ bar/.append style={ pattern=grid }, } \ganttbar[progress=100]{\raisebox{-0.075cm}{ Justificación } \ganttalignnewline \raisebox{0.15cm}{ \footnotesize{ \textit{ \textcolor{gray}{ Responsable(s): } \textcolor{html1}{ \textbf{ Gerardo}} }}}}{2024-09-12}{2024-09-17}
			\\ \ganttset{ bar/.append style={ pattern=north west lines }, } \ganttbar[progress=100]{\raisebox{-0.075cm}{ Alcances y limitaciones } \ganttalignnewline \raisebox{0.15cm}{ \footnotesize{ \textit{ \textcolor{gray}{ Responsable(s): } \textcolor{html1}{ \textbf{ Valentín}} }}}}{2024-09-12}{2024-09-17}
			\\ \ganttset{ bar/.append style={ pattern color=html2 }, group/.append style={ draw=black, fill=html2 }, milestone/.append style={ fill=html2 }, }
			\ganttset{ bar/.append style={ pattern=dots }, } \ganttgroup[progress=100]{Marco Teórico}{2024-09-18}{2024-09-24}
			\\ \ganttset{ bar/.append style={ pattern=dots }, } \ganttbar[progress=100]{\raisebox{-0.075cm}{ Análisis de requerimientos } \ganttalignnewline \raisebox{0.15cm}{ \footnotesize{ \textit{ \textcolor{gray}{ Responsable(s): } \textcolor{html2}{ \textbf{ Gerardo}} }}}}{2024-09-18}{2024-09-24}
			\\ \ganttset{ bar/.append style={ pattern=crosshatch dots }, } \ganttbar[progress=100]{\raisebox{-0.075cm}{ Diseño de vistas } \ganttalignnewline \raisebox{0.15cm}{ \footnotesize{ \textit{ \textcolor{gray}{ Responsable(s): } \textcolor{html2}{ \textbf{ Magaly}} , \textcolor{html2}{ \textbf{ Monserrat}} }}}}{2024-09-18}{2024-09-24}
			\\ \ganttset{ bar/.append style={ pattern=horizontal lines }, } \ganttbar[progress=100]{\raisebox{-0.075cm}{ Diseño conceptual } \ganttalignnewline \raisebox{0.15cm}{ \footnotesize{ \textit{ \textcolor{gray}{ Responsable(s): } \textcolor{html2}{ \textbf{ Aimar Jair}} }}}}{2024-09-18}{2024-09-24}
			\\ \ganttset{ bar/.append style={ pattern=vertical lines }, } \ganttbar[progress=100]{\raisebox{-0.075cm}{ Análisis de entidades } \ganttalignnewline \raisebox{0.15cm}{ \footnotesize{ \textit{ \textcolor{gray}{ Responsable(s): } \textcolor{html2}{ \textbf{ Valentín}} }}}}{2024-09-18}{2024-09-24}
			\\ \ganttset{ bar/.append style={ pattern=grid }, } \ganttbar[progress=100]{\raisebox{-0.075cm}{ Análisis funcional } \ganttalignnewline \raisebox{0.15cm}{ \footnotesize{ \textit{ \textcolor{gray}{ Responsable(s): } \textcolor{html2}{ \textbf{ Cristian}} }}}}{2024-09-18}{2024-09-24}
		\end{ganttchart}
		}
	\end{adjustbox}
\end{document}